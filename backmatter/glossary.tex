%% Le glossaire est un élément optionnel qui peut être utile pour expliquer des termes techniques.

\newglossaryentry{heig-vd}{
    name=HEIG-VD,
    description={Haute École d'Ingénierie et de Gestion du canton de Vaud}
}
\newglossaryentry{hes-so}{
    name=HES-SO,
    description={Haute École Supérieure de Suisse Occidentale}
}
\newglossaryentry{latex}{
    name=latex,
    description={Un langage et un système de composition de documents}
}
\newglossaryentry{maths}{
    name=mathematics,
    description={Les mathematiques sont ce que les mathématiciens fonts}
}
\newglossaryentry{python}{
    name=Python,
    description={Un langage de programmation interprété, orienté objet et de haut niveau}
}
\newglossaryentry{jerk}{
    name=jerk,
    description={Le jerk est le résultat de la dérivée de l'accélération}
}
\newglossaryentry{rviz}{
    name=Rviz,
    description={Un outil de visualisation pour le robot operating system (ROS)}
}
\newglossaryentry{docker}{
    name=Docker,
    description={Un outil de virtualisation léger qui permet de créer des conteneurs}
}
\newglossaryentry{conteneur}{
    name=Conteneur,
    description={Un conteneur est une unité standardisée de logiciel qui regroupe le code et toutes ses dépendances pour que l'application puisse s'exécuter rapidement et de manière fiable dans différents environnements de calcul}
}
\newglossaryentry{yaml}{
    name=YAML,
    description={YAML Ain't Markup Language : un format de sérialisation de données lisible par l'humain, utilisé pour la configuration et l'échange de données}
}

\newglossaryentry{hardware}{
    name=Hardware,
    description={L'ensemble des composants physiques d'un système informatique}
}
\newglossaryentry{software}{
    name=Software,
    description={L'ensemble des programmes et applications qui s'exécutent sur un système informatique}
}

\newglossaryentry{payload}{
    name=Payload,
    description={La charge utile, ou payload, désigne les données ou la charge transportée par un système, souvent en référence à des données envoyées ou reçues par un robot ou un véhicule autonome}
}

\newglossaryentry{aicaS}{
    name=AICA Studio,
    description={AICA Studio est un environnement de développement pour la robotique, basé sur ROS2, qui permet de créer des applications robotiques en utilisant des blocs fonctionnels}
}

\newglossaryentry{ufactory}{
    name=UFactory,
    description={UFactory est une entreprise spécialisée dans la fabrication de robots collaboratifs et d'outils robotiques}
}
\newglossaryentry{aica}{
    name=AICA,
    description={AICA est une entreprise qui développe des solutions logicielles pour la robotique, notamment AICA Studio}
}
\newglossaryentry{xarm6}{
    name=xArm6,
    description={Le xArm6 est un bras robotisé de la marque UFactory, conçu pour des applications industrielles et de recherche}
}
