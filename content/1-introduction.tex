\chapter{Introduction}

\section{Contexte}

Le projet de station de gravure laser intelligente a été initié par M. Chevalley et M. Costanzo lors d'un entretien. L'objectif est de démontrer les capacités des nouvelles technologies dans la robotique industrielle, en associant un bras robotisé et le logiciel AICA Studio produit par l'entreprise AICA. Cette station de gravure, innovante et polyvalente, peut être utilisée comme maquette de démonstration lors des portes ouvertes de l'école, d'événements publics ou pour la formation d'étudiants et de visiteurs intéressés par la robotique.

Ce projet met en avant l'intérêt pédagogique et la dimension didactique de l'automatisation moderne, tout en illustrant l'intégration de solutions logicielles et matérielles avancées.

\section{Problématique}

Les principaux défis de ce projet sont la synchronisation entre le bras robot et le laser ainsi que le contrôle sécurisé de la station. En effet, le laser et le robot sont des éléments qui peuvent être dangereux s'ils sont mal utilisés. Un effort particulier est nécessaire pour garantir la sécurité des utilisateurs et du public.

\section{Objectifs}

L'objectif principal de ce travail de Bachelor est de concevoir une maquette présentable à des événements publics et pédagogiques. Pour cela, il est nécessaire de poser certains sous-objectifs :

\begin{itemize}
    \item Analyser les risques liés à l'utilisation du robot avec du public
    \item Prendre en main le bras robot Ufactory Xarm6
    \item Prendre en main le logiciel AICA Studio
    \item Développer une interface graphique de contrôle de la station
    \item Synchroniser le laser et le robot
    \item Concevoir et intégrer les différents éléments électriques et électroniques de la station
    \item Assurer la présentation esthétique et didactique de la station
    \item Rendre le système fiable et facile d'utilisation
\end{itemize}

Ces sous-objectifs posent un cadre clair pour le développement du projet, en mettant l'accent sur la sécurité, l'innovation et la valorisation pédagogique.


