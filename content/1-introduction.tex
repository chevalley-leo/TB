\chapter{Introduction}

\section{Contexte}

Le projet de station de gravue laser intelligente a été initié par M.Chevalley et M.Costanzo pendant un entretient. L'idée est de démontrer les capacités des nouvelles technologies dans la robotique industrielle en utilisant un bras robot et le logiciel AICA Studio proudit par l'entreprise AICA. Cette station de gravure peut etre utilisée comme maquette de démonstration lors des portes ouvertes de l'école ou tout autre événement.

\section{Problématique}

Les principaux défis de ce projet sont la synchronisation entre le bras robot et le laser ainsi que le contrôle sécurisé de la sation. En effet, le laser et le robot sont des éléments qui peuvent être dangereux si ils sont mal utilisés. Un effort particulier est necessaire pour garantir la sécurité des utilisateurs.

\section{Objectifs}

L'objectif principal de ce travail de Bachelor est de réussir à concevoir une maquette présentable à des événements publiques. Pour cela, il est nécessaire de poser certains
sous objectifs:

\begin{itemize}
    \item Analyser les risques liés à l'utilisation du robot avec du public
    \item Prendre en main le bras robot Ufactory Xarm6
    \item Prendre en main le logiciel AICA Studio
    \item Développer une interface graphique de contrôle de la station
    \item Synchroniser le laser et le robot
    \item Concevoir et intégrer les différents éléments électrique et électronique de la station
    \item Assurer la présentation estétique et didactique de la station
    \item Rendre le système fiable et facile d'utilisation
\end{itemize}

Ces sous objectifs posent un cadre clair pour le développement du projet.


