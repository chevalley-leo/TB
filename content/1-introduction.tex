\chapter{Introduction}
\label{chap:intro}


\section{Contexte}

Le projet de station de gravure laser intelligente a été initié par M. Chevalley et M. Costanzo lors d'un entretien. L'objectif est de démontrer les capacités des nouvelles technologies dans la robotique industrielle, en associant un bras robotisé et le logiciel \gls{aicaS}  produit par l'entreprise \gls{aica}. Cette station de gravure, innovante et polyvalente, peut être utilisée comme maquette de démonstration lors des portes ouvertes de l'école, d'événements publics ou pour la formation d'étudiants et de visiteurs intéressés par la robotique.

Ce projet met en avant l'intérêt pédagogique et la dimension didactique de l'automatisation moderne, tout en illustrant l'intégration de solutions logicielles et matérielles avancées.

\section{Problématique}

Le principal défi de ce projet réside dans le fait qu'il s'agit d'une maquette de démonstration destinée à être présentée lors d'événements publics ou pédagogiques. Il est donc essentiel que le système soit suffisamment fiable, robuste et simple d'utilisation pour fonctionner sans intervention technique fréquente, même dans des conditions variées et devant des utilisateurs non spécialistes. La synchronisation entre le bras robot et le laser, la reconnaissance automatique des pièces à graver et la gestion conjointe des différents éléments doivent être conçues pour garantir un fonctionnement fluide et sécurisé, tout en offrant une expérience convaincante et pédagogique.


\section{Objectifs}

L'objectif principal de ce travail de Bachelor est de concevoir une maquette présentable lors d'événements publics et pédagogiques. Pour cela, il est nécessaire de poser les sous-objectifs suivants :

\begin{itemize}
    \item Prendre en main le bras robot \gls{ufactory} \gls{xarm6}
    \item Prendre en main le logiciel \gls{aicaS}
    \item Développer une interface graphique de contrôle de la station
    \item Synchroniser le laser et le robot
    \item Concevoir et intégrer les différents éléments électriques et électroniques de la station
    \item Analyser les risques liés à l'utilisation du robot avec du public
    \item Assurer la présentation esthétique et didactique de la station
    \item Rendre le système fiable et facile d'utilisation
\end{itemize}

Ces sous-objectifs posent un cadre clair pour le développement du projet, en mettant l'accent sur la sécurité, l'innovation et la valorisation pédagogique.


