\chapter{Analyse}

\index{état de l'art}
\index{méthode}


\section{État de l'art}

\section{Analyse des risques}

\section{Prise en main du bras robot}
La prise en main du bras robot Ufactory Xarm6 a été une des première étape du projet. Le but était de se familiariser avec les commandes et de comprendre les limites physiques et logicielles du robot.

\subsection{Mise en marche du robot et connexion}
Avec le manuel utilisateur fourni par Ufactory \cite{UserManual}, il a été possible de comprendre et prendre en main le robot. La première étape après la mise sous tension à été la connexion ethernet entre le contrôleur et le logiciel UFACTORY-Studio. La procédure est très simple et ne demande pas de configuration particulière.

\subsection{UFACTORY-Studio}
Le logiciel UFACTORY-Studio est utile pour configurer la position initiale du robot, les limites de mouvements, la force maximale de chaque articulation et les limites de distances à ne pas dépasser.

En plus de cela, le logiciel offre un contrôle complet sur les entrées/sorties (appelées à partir de maintenant \gls{io}), les axes, la vitesse de déplacement et l'ouverture/fermeture de la pince. Il est également possible de créer des programmes de mouvement en utilisant le langage  de programation Python ou un langage visuel propriétaire basé sur le Python. Le langage visuel reste très basique et ne permet pas de créer des programmes complexes qui sortent du cadre de la cinématique du robot.

\subsection{Python SDK}
Pour s'affranchir des limitations du logiciel UFACTORY-Studio, il est possible d'utiliser le SDK Python \cite{PythonSDK} fourni par Ufactory. Ce SDK permet de contrôler le robot de manière plus avancée et de créer des programmes plus complexes. Le langage de base de contrôle du robot étant le Python, il est possible d'utiliser n'importe quel IDE Python pour développer des programmes. La documentation du SDK étant très complète avec de nombreux exemples, la compréhension et la prise en main du SDK se sont faites rapidement. De plus, les contrôles de cinématique sont plus "bas niveau" ce qui offre plus de flexibilité notamment au niveau de l'accéleration, du \gls{jerk} et des opérations sur les positions.

\subsection{Entrées/sorties}
Le contrôleur du bras robot offre plusieurs types d'IO:

\definecolor{SafetyColor}{HTML}{FFDDC1}
\definecolor{PowerColor}{HTML}{C1E1FF}
\definecolor{ConfigInputColor}{HTML}{D4FFC1}
\definecolor{DigitalInputColor}{HTML}{FFD1DC}
\definecolor{ConfigOutputColor}{HTML}{E1C1FF}
\definecolor{DigitalOutputColor}{HTML}{C1FFE1}
\definecolor{RS485Color}{HTML}{FFC1E1}
\definecolor{AnalogColor}{HTML}{E1FFC1}

\begin{tcolorbox}[colframe=black, colback=SafetyColor, title=Safety]
\textbf{Connexions :} GND | EI0 | GND | EI1 | GND | SI0 | GND | SI1
\end{tcolorbox}

\begin{tcolorbox}[colframe=black, colback=PowerColor, title=Power]
\textbf{Connexions :} PWR | 24V-IN | GND | RI0 | NC | GND | ON | OFF
\end{tcolorbox}

\begin{tcolorbox}[colframe=black, colback=ConfigInputColor, title=Configurable Inputs]
\textbf{Connexions :} GND | CI0 | CI1 | CI2 | CI3 | GND | CI4 | CI5 |CI6 | CI7
\end{tcolorbox}

\begin{tcolorbox}[colframe=black, colback=DigitalInputColor, title=Digital Inputs]
\textbf{Connexions :} GND |  DI0 | DI1 | DI2 | DI3 | GND | DI4 | DI5 | DI6 | DI7
\end{tcolorbox}

\begin{tcolorbox}[colframe=black, colback=ConfigOutputColor, title=Configurable Outputs]
\textbf{Connexions :} 24V | CO0 | CO1 | CO2 | CO3 | 24V | CO4 | CO5 | CO6 | CO7
\end{tcolorbox}

\begin{tcolorbox}[colframe=black, colback=DigitalOutputColor, title=Digital Outputs]
\textbf{Connexions :} 24V | DO0 | DO1 | DO2 | DO3 | 24V | DO4 | DO5 | DO6 | DO7
\end{tcolorbox}

\begin{tcolorbox}[colframe=black, colback=RS485Color, title=RS485]
\textbf{Connexions :} 24V | 24V | M\_A | M\_B | GND | L\_A | L\_B | GND
\end{tcolorbox}

\begin{tcolorbox}[colframe=black, colback=AnalogColor, title=Analog]
\textbf{Connexions :} GND | AI0 | GND | AI1 | GND | AO0 | GND | AO1
\end{tcolorbox}

Ces IO premettent la communication avec des éléments externes. Néamoins, après avoir mesuré la fréquence de rafraichissement de ces IO, il s'avère que le taux de mise à jour est trop faible pour permettre la création d'un signal \gls{pwm}. Ce signal PWM est nécessaire pour le contrôle du laser qui sera expliqué plus tard dans ce rapport. De ce fait, une sortie analogique est donc utilisée avec un convertisseur analogique/PWM pour permettre un contrôle précis du laser.

\section{Prise en main de AICA Studio}
AICA Studio est un logiciel qui simplifie l'intégration et la programmation de robots industriels via des blocs de fonctions prédéfinis. Il est compatible avec le bras robot Ufactory Xarm6 et permet de créer des programmes de manière visuelle, sans nécessiter de compétences avancées en programmation. Le logiciel s'appuie sur \gls{ros2}, qui gère les aspects bas-niveau du contrôle et de la communication, tandis que AICA, en tant que couche supérieure, propose une interface simplifiée pour la gestion des tâches complexes, telles que la planification de trajectoires ou l'intégration d'opérations avancées comme la détection de collisions ou le contrôle du robot avec une l'interface \gls{rviz}.

\subsection{Installation de AICA Studio}
L'installation de AICA Studio est, contrairement à UFACTORY-Studio, un peu plus complexe. Il est nécessaire d'utiliser un système d'exploitation Linux ou Posix. Le guide d'installation \cite{AICADocs} est complet et facile à suivre. Cependant, il est tout de même nécessaire d'avoir quelques connaissance de base en ligne de commandes et en gestion de systèmes Linux. L'avantage de cette installation est qu'elle permet de créer un environnement complet en incluant dans un \gls{conteneur} \gls{docker} tous les outils nécessaire au bon fonctionnement de l'application.

\subsection{Démarrage de AICA Studio}
