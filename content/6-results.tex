\chapter{Résultats}

\section{Fonctionnement global}

L'ensemble du processus, de la création du dessin sur l'écran tactile jusqu'à la remise de la pièce gravée à l'utilisateur, a pu être testé dans différentes conditions. Le système fonctionne de manière fluide lorsque toutes les conditions sont réunies : la détection de la pièce est correcte, le dessin est bien transmis et la gravure s'effectue sans incident. Plusieurs scénarios ont été réalisés, notamment la gravure du logo par défaut, de dessins libres et de formes géométriques simples.

\section{Qualité de la gravure}

La qualité de la gravure dépend fortement de la précision du dessin initial et du positionnement de la pièce. Dans la majorité des cas, le motif est reproduit fidèlement sur la pièce. Quelques défauts peuvent apparaître si la pièce est mal positionnée ou si le dessin est mal interprêté par le programme.


\section{Performance de la détection}

La détection de la pièce par la caméra fonctionne correctement dans la plupart des cas. Cependant, certains facteurs comme la couleur de la peau de l'utilisateur ou un éclairage trop variable peuvent perturber l'algorithme. Sur 20 essais, 17 ont réussi, soit un taux de réussite de 85\%. Les échecs sont principalement dus à des conditions lumineuses défavorables ou à une confusion entre la couleur de la pièce et celle de la main. Dans le cas de l'utilisation de la machine avec des gants, le taux de réussite s'élève à 95\%, car la couleur du gant est différente de celle de la pièce.

\section{Robustesse et sécurité}

Le système réagit bien en cas d'erreur : si la pièce n'est pas détectée ou mal prise, le robot retourne en position d'attente sans forcer la prise. Les tests d'arrêt d'urgence ont montré que la sécurité de l'utilisateur est assurée, le robot s'arrêtant immédiatement en cas de problème.

\section{Limites observées}

Certaines limites ont été identifiées lors des tests :
\begin{itemize}
    \item La détection de la pièce reste sensible à la couleur et à l'éclairage.
    \item Les dessins ne sont pas toujours bien reproduits.
    \item Le temps de gravure peut varier selon la complexité du motif.
    \item La taille des pièces est fixe et ne peut être modifiée qu'avec une grosse mise à jour logicielle
\end{itemize}

\section{Synthèse des essais}

\begin{table}[H]
    \centering
    \begin{tabular}{|c|c|c|c|}
        \hline
        Essai & Détection & Gravure & Remise \\
        \hline
        1 & OK & OK & OK \\
        2 & OK & Défaut & OK \\
        3 & Échec & -- & -- \\
        4 & OK & OK & OK \\
        5 & OK & OK & OK \\
        ... & ... & ... & ... \\
        20 & OK & OK & OK \\
        \hline
    \end{tabular}
    \caption{Synthèse des essais réalisés}
    \label{tab:synthese_essais}
\end{table}

\section{Retours utilisateurs}

Les utilisateurs ayant testé la station ont apprécié la simplicité de l'interface tactile et la rapidité du processus. Quelques remarques ont été faites concernant la nécessité de bien positionner la pièce. Globalement, l'expérience utilisateur est jugée positive et amusante.

