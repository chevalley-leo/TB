\chapter{Implémentation}


\index{expérience}
Vous devez décrire le déroulement de vos expériences, en précisant le cadre dans lequel elles ont été menées ainsi que les outils et technologies utilisés. Ce chapitre doit permettre au lecteur de comprendre non seulement ce que vous avez fait, mais aussi pourquoi vous avez opté pour cette approche.

Exposez de manière structurée la conception globale de votre projet en illustrant vos explications par des schémas, des diagrammes ou des modèles visuels -- tels que des diagrammes de classes, de flux ou des organigrammes -- afin de clarifier votre démarche.

Que ce soit pour une architecture logicielle, une conception électronique ou une analyse de données, chaque aspect de votre travail doit être présenté de façon logique et détaillée. Il est essentiel de justifier vos choix en soulignant leur pertinence au regard des objectifs fixés et de la problématique posée.

Vous êtes libre de structurer ce chapitre en plusieurs sections distinctes, en fonction de la nature de votre projet. L'objectif est d'offrir au lecteur une vision claire de la manière dont vous avez développé votre travail, en mettant en lumière non seulement les actions entreprises, mais également les décisions qui les ont motivées. Votre implémentation doit refléter une réflexion méthodique et rigoureuse, guidée par la problématique et les objectifs initiaux.

\section{Implémentation des différents éléments}

La maquette finale de la graveuse laser possède plusieurs éléments qui lui permettent de fonctionner. Ces éléments sont :
