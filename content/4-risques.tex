\chapter{Risques}
\label{chap:risques}

Après l'implémentation de la station de gravure laser intelligente, il est essentiel de procéder à une analyse des risques. Cette étape permet d'identifier les dangers potentiels liés à l'utilisation du robot et du laser dans cet environement spécifique. Grâce à cette analyse, des mesures physiques et logicielles de prévention et de maîtrise des risques peuvent être mises en place.

\section{Analyse des risques}
L'utilisation d'une station de gravure laser intelligente avec un bras robotisé en environnement public comporte plusieurs risques qu'il convient d'identifier, d'évaluer et de maîtriser. L'analyse suivante s'appuie sur le contexte du projet, les éléments matériels et logiciels utilisés, ainsi que les objectifs de sécurité et de fiabilité.

\subsection{Risques identifiés}
\begin{itemize}
    \item \textbf{Risque laser} : Le laser utilisé pour la gravure présente un danger pour la vue et la peau en cas d'exposition accidentelle. Un mauvais contrôle logiciel ou une défaillance matérielle pourrait entraîner l'activation du laser hors de la zone prévue.
    \item \textbf{Risque robotique} : Le bras robot Ufactory Xarm6, bien que collaboratif, reste potentiellement dangereux. Il peut provoquer des blessures par collision ou pincement si un utilisateur s'approche trop près pendant le fonctionnement. Une sous section est dédiée à l'analyse de risques concernant le robot.
    \item \textbf{Risque électrique} : La station comporte plusieurs alimentations électriques (robot, laser, écran tactile) pouvant présenter un danger en cas de défaut d'isolation ou de manipulation inappropriée. Notamment au niveau des alimentations 230V des différents éléments.
    \item \textbf{Risque logiciel et synchronisation} : Une erreur dans la synchronisation entre le robot et le laser pourrait activer le laser au mauvais moment ou déplacer le robot de façon imprévue.
    \item \textbf{Risque de mauvaise détection} : Une défaillance de la caméra ou une erreur dans le traitement des données pourrait entraîner une mauvaise localisation de la pièce à graver, provoquant des collisions ou des gravures hors zone.
    \item \textbf{Risque d'utilisation publique} : En contexte de démonstration, le public peut adopter des comportements imprévisibles (toucher, interférer, etc.), augmentant les risques d'accident.
\end{itemize}

\subsection{Tableau des risques}

\rowcolors{2}{white}{gray!10}
\begin{table}[H]
    \centering
    \renewcommand{\arraystretch}{1.4}
    \begin{tabular}{|p{3.5cm}|>{\centering\arraybackslash}m{2.2cm}|>{\centering\arraybackslash}m{2.2cm}|>{\centering\arraybackslash}m{2.2cm}|p{3.5cm}|}
        \hline
        \textbf{Risque} & \textbf{Sévérité} & \textbf{Fréquence} & \textbf{Importance} & \textbf{Description} \\
        \hline
        Collision avec un utilisateur & \cellcolor{orange!60}Moyenne & \cellcolor{yellow!60}Faible & \cellcolor{orange!60}Moyenne & Blessure possible en cas de contact entre le bras robot et une personne. \\
        \hline
        Pincement lors du mouvement & \cellcolor{green!60}Très Faible & \cellcolor{yellow!60}Faible & \cellcolor{yellow!60}Faible & Risque de pincement des doigts ou de la main lors des mouvements du robot, notamment avec la pince. \\
        \hline
        Mauvaise préhension de la pièce & \cellcolor{yellow!60}Faible & \cellcolor{red!60}Élevée & \cellcolor{orange!60}Moyenne & Le robot peut mal saisir la pièce, entraînant des erreurs de positionnement et des comportements imprévus. \\
        \hline
        Mouvement imprévu suite à une erreur logicielle & \cellcolor{yellow!60}Faible & \cellcolor{green!60}Très faible & \cellcolor{yellow!60}Faible & Un bug logiciel peut entraîner un déplacement non anticipé du robot. \\
        \hline
        Défaillance du système de sécurité & \cellcolor{red!60}Élevée & \cellcolor{green!60}Très faible & \cellcolor{orange!60}Moyenne & Si les dispositifs de sécurité ne fonctionnent pas, le robot peut devenir dangereux. \\
        \hline
    \end{tabular}
    \caption{Tableau d'analyse des risques robotiques pour le bras Ufactory Xarm6 dans le cadre de l'utilisation de la station de gravure laser intelligente.}
    \label{tab:risques_robotique}
\end{table}

Cette analyse permet de mettre en place des mesures de sécurité pour la station, en anticipant les principaux scénarios à risque.

\subsection{Mesures de prévention et de maîtrise}
\begin{itemize}
    \item Mise en place de protections physiques (capot, barrières) autour de la zone de gravure. (à mettre en place)
    \item Activation du laser uniquement lorsque le robot est en position sécurisée et la zone libre.
    \item Surveillance logicielle de la synchronisation robot/laser et gestion des erreurs critiques.
    \item Validation logicielle de la détection de pièce et limitation des mouvements du robot en cas d'incertitude.
    \item Présence d'un opérateur formé lors des démonstrations publiques.
    \item Ajout d'une vérification de prise de pièce via une caméra.
\end{itemize}

Ces mesures permettent de réduire significativement les risques identifiés, en garantissant un fonctionnement sécurisé et fiable de la station de gravure laser intelligente.

L'ajout de ces sécurités a été fait au fur et à mesure du développement du projet. L'ajout de la verification de la prise de pièce via une caméra a été implémenté en dernier suite à une réflexion sur les potentielles améliorations et fiabilisations du système. En effet, la vérification de la prise de pièce permet de s'assurer que le robot a bien saisi la pièce avant de se déplacer vers la zone de gravure, réduisant ainsi les risques que la graveuse laser active le laser dans le vide.
