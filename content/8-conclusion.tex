\chapter{Conclusion}

Ce projet de maquette de gravure laser intelligente a été une aventure technique et humaine particulièrement enrichissante. L'objectif initial était ambitieux : réussir à automatiser l'ensemble du processus, du dessin sur écran tactile jusqu'à la remise de la pièce gravée, en passant par la détection, la manipulation robotisée et la gravure proprement dite. Chaque étape a apporté son lot de défis, de doutes, mais aussi de satisfactions lorsque les différentes briques se sont enfin mises à fonctionner ensemble.

L'intégration du bras robot Ufactory Xarm6, de la graveuse Creality Falcon, de la caméra Intel D435 et de l'écran tactile n'a pas été de tout repos. Il a fallu jongler avec les contraintes matérielles, les limitations logicielles, et parfois improviser pour contourner des problèmes inattendus. La détection de la pièce, en particulier, s'est révélée plus complexe que prévu, notamment à cause des variations de couleur et de lumière, ou encore des différences entre les pièces et l'environnement. Chaque obstacle surmonté a permis de mieux comprendre les interactions entre les différents éléments et d'affiner la solution.

Au final, le système obtenu est non seulement fonctionnel, mais aussi fiable et agréable à utiliser. La qualité de gravure est satisfaisante, la sécurité de l'utilisateur est assurée, et l'ergonomie de l'interface tactile a été appréciée lors des démonstrations. Il reste bien sûr des axes d'amélioration, notamment pour rendre la détection encore plus robuste, accélérer certains cycles, ou enrichir l'expérience utilisateur avec de nouvelles fonctionnalités. Mais la base est solide et le potentiel d’évolution reste important.

Ce projet a permis de mettre en pratique de nombreuses compétences, tant sur le plan technique (robotique, vision, automatisation, programmation) que sur le plan organisationnel (gestion de projet, adaptation, communication). Il a aussi été l'occasion de collaborer avec différents interlocuteurs, de partager des idées et de confronter des points de vue, ce qui a largement contribué à la réussite de l'ensemble.

En outre, ce projet ouvre la voie à de nombreuses perspectives d’évolution. À l’avenir, il serait envisageable d’intégrer des algorithmes de détection plus avancés, basés sur l’intelligence artificielle, afin d’améliorer la robustesse face aux variations d’environnement ou à la diversité des pièces. L’ajout de nouveaux capteurs, comme une caméra couleur haute résolution ou des capteurs de force sur la pince, permettrait d’affiner encore la précision des manipulations et d’augmenter la sécurité.

Par ailleurs, l’ergonomie de l’interface utilisateur pourrait être enrichie par de nouvelles fonctionnalités, telles que la personnalisation avancée des motifs, la gestion de bibliothèques de dessins, ou encore l’intégration d’un retour vocal ou visuel pour guider l’utilisateur tout au long du processus.

En conclusion, cette maquette de gravure laser intelligente n'est pas seulement un démonstrateur technique : c'est le fruit d'un travail passionné, d'une volonté d'aller au bout des choses et d'une envie de proposer une solution innovante, à la fois pédagogique et ouverte sur l'industrie du futur.

\vspace*{2cm}
\begin{flushright}
\makeatletter\@author\makeatother

\begin{minipage}{5cm}
    \printsignature
\end{minipage}
\end{flushright}