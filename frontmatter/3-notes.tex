\chapter*{Préface}
\addcontentsline{toc}{chapter}{Préface}

Ce document présente le travail de Bachelor réalisé au sein de la Haute École d'Ingénierie et de Gestion du Canton de Vaud (HEIG-VD), dans le cadre de la filière Génie électrique.

Le projet décrit porte sur le développement d'une station de gravure laser intelligente, intégrant un bras robotisé, un système de vision 3D et une interface de contrôle. Il s'inscrit dans une démarche de démonstration technologique visant à illustrer les possibilités offertes par la robotique collaborative et l'automatisation industrielle.

La rédaction de ce rapport suit les standards académiques en vigueur à la HEIG-VD et repose sur une structure formelle visant à présenter de manière claire et rigoureuse les différentes étapes du projet, de l’analyse initiale à la mise en œuvre finale.

Les travaux ont été réalisés sous la supervision de M.Costanzo et avec le soutien de M. Bloesch et de l'entreprise AICA.

