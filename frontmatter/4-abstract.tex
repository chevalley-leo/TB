\begin{abstract}

Ce projet de Bachelor a consisté à développer une station de gravure laser automatisée, combinant un bras robotisé UFactory Xarm6, une caméra de vision 3D Intel RealSense D435 et une graveuse laser Creality Falcon. L’objectif principal était de permettre la détection automatique de pièces dans l’espace de travail à l’aide de la caméra, et de générer dynamiquement un parcours de gravure adapté à chaque position détectée.

L’environnement de développement AICA Studio a été utilisé pour créer les blocs fonctionnels nécessaires à l’intégration des différents composants. Un bloc personnalisé a été conçu pour la détection 3D de la pièce via la caméra, et un autre pour piloter la graveuse laser à l’aide de commandes G-code générées automatiquement à partir de fichiers DXF.

Les résultats obtenus démontrent la faisabilité d’un tel système en environnement industriel, avec une reconnaissance de pièce fiable et une gravure précise adaptée à la position réelle de la pièce. Cette approche offre des perspectives intéressantes pour des applications de production flexibles ou de prototypage rapide.

\asterism

This Bachelor project involved the development of an automated laser engraving station, combining a UFactory Xarm6 robotic arm, an Intel RealSense D435 3D vision camera, and a Creality Falcon laser engraver. The main goal was to enable automatic detection of workpieces in the workspace using the camera, and to dynamically generate an engraving path tailored to each detected position.

The development environment AICA Studio was used to create the required functional blocks. A custom block was implemented for 3D part detection via the camera, and another for controlling the laser engraver using G-code commands automatically generated from DXF files.

The results demonstrate the feasibility of such a system in an industrial context, with reliable part recognition and accurate engraving adapted to the real-world position of the workpiece. This approach offers promising prospects for flexible manufacturing and rapid prototyping applications.

\end{abstract}
